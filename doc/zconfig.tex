\documentclass{howto}

\title{ZConfig Package Reference}

%\date{\today}
%\release{0.00}

\author{Zope Corporation}
\authoraddress{
    Lafayette Technology Center\\
    513 Prince Edward Street\\
    Fredericksburg, VA 22401\\
    \url{http://www.zope.com/}
}

\begin{document}
\maketitle

\begin{abstract}
\noindent
This document describes the syntax and API used in configuration files
for components of a Zope installation written by Zope Corporation.
\end{abstract}

\tableofcontents


\section{Introduction \label{intro}}

Zope uses a common syntax and API for configuration files designed for
software components written by Zope Corporation.  Third-party software
which is also part of a Zope installation may use a different syntax,
though any software is welcome to use the syntax used by Zope
Corporation.  Any software written in Python is free to use the
\module{ZConfig} software to load such configuration files in order to
ensure compatibility.  This software is covered by the Zope Public
License, version 2.0.

The \module{ZConfig} package has been tested with Python 2.1 and 2.2.
It only relies on the Python standard library.


\section{Configuration Syntax \label{syntax}}


\section{\module{ZConfig} --- Basic configuration support}

\declaremodule{}{ZConfig}
\modulesynopsis{Configuration package}

The main \module{ZConfig} package exports a single function:

\begin{funcdesc}{load}{url}
  Load and return a configuration from a URL or pathname given by
  \var{url}.  \var{url} may be a URL, absolute pathname, or relative
  pathname.
\end{funcdesc}


\section{\module{ZConfig.Context} --- Application context}

\declaremodule{}{ZConfig.Context}
\modulesynopsis{Application context}


\section{\module{ZConfig.Config} --- Section objects}

\declaremodule{}{ZConfig.Config}
\modulesynopsis{Standard section objects}


The \module{ZConfig.Config} module provides implementations of the
standard key-value section.  There are two implementations: the basic
implementation used for ``internal'' sections, and a subclass that
provides additional support for the \keyword{import} statement (used
for the top level of a configuration and for imported resources).

\begin{classdesc}{Configuration}{type, name, url}
  A typed section with an optional name.  The type is given by the
  \var{type} argument, and the URL the configuration is loaded from is
  given by \var{url}.  Both \var{type} and \var{url} must be strings.
  The optional name of the section is given by \var{name}; if there is
  no name, \var{name} should be \code{None}.
\end{classdesc}

\begin{classdesc}{ImportingConfiguration}{type, name, url}
  A subclass of \class{Configuration} which supports the context
  needed to support the \keyword{import} directive.  This class
  differs from the base class in that it offers an additional method
  and changes the lookup semantics of the \method{get()} method.
\end{classdesc}

\class{Configuration} objects provide the following methods to
retrieve values from the section:

\begin{methoddesc}[Configuration]{get}{key\optional{, default}}
  Returns the value for \var{key} as a string; a value from the
  delegate section is used if needed.  If there is no value for
  \var{key}, returns \var{default}.
\end{methoddesc}

\begin{methoddesc}[Configuration]{getbool}{key\optional{, default}}
  Returns the value for \var{key} as a \class{bool}.  If there is no
  value for \var{key}, returns \var{default}.  Conversions to
  \class{bool} are case-insensitive; the strings \code{true},
  \code{yes}, and \code{on} cause \code{True} to be returned; the
  strings \code{false}, \code{no}, and \code{off} generate
  \code{False}.  All other strings cause \exception{ValueError} to be
  raised.
\end{methoddesc}

\begin{methoddesc}[Configuration]{getfloat}{key\optional{,
        default\optional{, min\optional{, max}}}}
  Return the value for \var{key} as a float.  If there is no
  value for \var{key}, returns \var{default}.  If the value cannot
  be converted to a float, \exception{ValueError} is raised.  If
  \var{min} is given and the value is less than \var{min}, or if
  \var{max} is given and the value is greater than \var{max},
  \exception{ValueError} is raised.  No range checking is performed if
  neither \var{min} nor \var{max} is given.
\end{methoddesc}

\begin{methoddesc}[Configuration]{getint}{key\optional{,
        default\optional{, min\optional{, max}}}}
  Return the value for \var{key} as an integer.  If there is no
  value for \var{key}, returns \var{default}.  If the value cannot
  be converted to an integer, \exception{ValueError} is raised.  If
  \var{min} is given and the value is less than \var{min}, or if
  \var{max} is given and the value is greater than \var{max},
  \exception{ValueError} is raised.  No range checking is performed if
  neither \var{min} nor \var{max} is given.
\end{methoddesc}

\begin{methoddesc}[Configuration]{items}{}
  Return a list of key-value pairs from this section, including any
  available from the delegate section.
\end{methoddesc}

\begin{methoddesc}[Configuration]{keys}{}
  Return a list of keys from this section, including any available
  from the delegate section.
\end{methoddesc}


The following methods are used to modify the values defined in a
section:

\begin{methoddesc}[Configuration]{addValue}{key, value}
  Add the key \var{key} with the value \var{value}.  If there is
  already a value for \var{key}, \exception{ConfigurationError} is
  raised.
\end{methoddesc}

\begin{methoddesc}[Configuration]{setValue}{key, value}
  Set the value for \var{key} to \var{value}.  If there is already a
  value for \var{key}, it is replaced.  \var{key} and \var{value} must
  be strings.
\end{methoddesc}


The following methods are used in retrieving and managing sections:

\begin{methoddesc}[Configuration]{addChildSection}{section}
  Add a section that is a child of the current section.
\end{methoddesc}

\begin{methoddesc}[Configuration]{addNamedSection}{section}
  Add a named section to this section's context.  This is only used to
  add sections that are descendents but not children of the current
  section.
\end{methoddesc}

\begin{methoddesc}[Configuration]{getChildSections}{}
  Returns a sequence of all child sections, in the order in which they
  were added.
\end{methoddesc}

\begin{methoddesc}[Configuration]{getSection}{type\optional{, name}}
  Returns a single typed section.  The type of the retrieved section
  is given by \var{type}.  If \var{name} is given and not \code{None},
  the name of the section must match \var{name}.  If there is no
  section matching in both name and type, \exception{KeyError} is
  raised.  If \var{name} is not given or is \code{None}, there must be
  exactly one child section of type \var{type}; that section is
  returned.  If there is more than one section of type \var{type},
  \exception{ConfigurationConflictingSectionError} is raised.  If
  there is no matching section and a delegate is available, it's
  \method{getSection()} method is called to provide the return value,
  otherwise \code{None} is returned.
\end{methoddesc}

Delegation is supported by one additional method:

\begin{methoddesc}[Configuration]{setDelegate}{section}
  Set the delegate section to \var{section} if not already set.  If
  already set, raises \exception{ConfigurationError}.
\end{methoddesc}


The \class{ImportingConfiguration} subclass offers an additional
method, normally not needed by applications, but possibly useful for
alternate configuration parsers.  Objects returned by the
context object's \method{createToplevelSection()} method need to
support this interface.

\begin{methoddesc}[ImportingConfiguration]{addImport}{section}
  Add a configuration generated from an import.
\end{methoddesc}


\section{\module{ZConfig.Common} --- Exceptions}

\declaremodule{}{ZConfig.Common}
\modulesynopsis{Exceptions}

\begin{excdesc}{ConfigurationError}
  Base class for exceptions specific to the \module{ZConfig} package.
  All instances provide a \member{message} attribute that describes
  the specific error.
\end{excdesc}

\begin{excdesc}{ConfigurationSyntaxError}
  Exception raised when a configuration source does not conform to the
  allowed syntax.  In addition to the \member{message} attribute,
  exceptions of this type offer the \member{url} and \member{lineno}
  attributes, which provide the URL and line number at which the error
  was detected.
\end{excdesc}

\begin{excdesc}{ConfigurationTypeError}
\end{excdesc}

\begin{excdesc}{ConfigurationMissingSectionError}
\end{excdesc}

\begin{excdesc}{ConfigurationConflictingSectionError}
\end{excdesc}


\section{\module{ZConfig.ApacheStyle} --- Apache-style parser}

\declaremodule{}{ZConfig.ApacheStyle}
\modulesynopsis{Parser for Apache-style configurations}

The \module{ZConfig.ApacheStyle} module implements the configuration
parser.  Most applications will not need to use this module directly.

This module provides a single function:

\begin{funcdesc}{Parse}{file, context, section, url}
  Parse text from the open file object \var{file}.  The application
  context is given as \var{context}, the section object that values
  and sections should be added to is given as \var{section}, and the
  URL of the resource being parsed is given in \var{url}.
\end{funcdesc}


\section{\module{ZConfig.Interpolation} --- String interpolation}

\declaremodule{}{ZConfig.Interpolation}
\modulesynopsis{Shell-style string interpolation helper}

This module provides a basic substitution facility similar to that
found in the Bourne shell (\program{sh} on most \UNIX{} platforms).  

The replacements supported by this module include:

\begin{tableiii}{l|l|c}{code}{Source}{Replacement}{Notes}
  \lineiii{\$\$}{\code{\$}}{(1)}
  \lineiii{\$\var{name}}{The result of looking up \var{name}}{(2)}
  \lineiii{\$\{\var{name}\}}{The result of looking up \var{name}}{}
\end{tableiii}

\noindent
Notes:
\begin{description}
  \item[(1)]  This is different from the Bourne shell, which uses
              \code{\textbackslash\$} to generate a \character{\$} in
              the result text.  This difference avoids having as many
              special characters in the syntax.

  \item[(2)]  Any character which immediately follows \var{name} may
              not be a valid character in a name.
\end{description}

In each case, \var{name} is a non-empty sequence of alphanumeric and
underscore characters not starting with a digit.  If there is not
a replacement for \var{name}, it is replaced with an empty string.

For these functions, the \var{mapping} argument can be a \class{dict},
or any type that supports the \method{get()} method of the mapping
protocol.

\begin{funcdesc}{interpolate}{s, mapping}
  Interpolate values from \var{mapping} into \var{s}.  Replacement
  values are copied into the result without further interpretation.
  Raises \exception{InterpolationSyntaxError} if there are malformed
  constructs in \var{s}.
\end{funcdesc}

\begin{funcdesc}{get}{mapping, name\optional{, default}}
  Return the value for \var{name} from \var{mapping}, interpolating
  values recursively if needed.  If \var{name} cannot be found in
  \var{mapping}, \var{default} is returned without being
  interpolated.
  Raises \exception{InterpolationSyntaxError} if there are malformed
  constructs in \var{s}, or \exception{InterpolationRecursionError} if
  any name expands to include a reference to itself either directly or
  indirectly.
\end{funcdesc}

The following exceptions are defined:

\begin{excdesc}{InterpolationError}
  Base class for errors raised by the \module{ZConfig.Interpolation}
  module.  Instances provide the attributes \member{message} and
  \member{context}.  \member{message} contains a description of the
  error.  \member{context} is either \code{None} or a list of names
  that have been looked up in the case of nested interpolation.
\end{excdesc}

\begin{excdesc}{InterpolationSyntaxError}
  Raised when interpolation source text contains syntactical errors.
\end{excdesc}

\begin{excdesc}{InterpolationRecursionError}
  Raised when a nested interpolation is recursive.  The
  \member{context} attribute will always be a list for this
  exception.  An additional attribute, \member{name}, gives the name
  for which an recursive reference was detected.
\end{excdesc}


\subsection{Examples}

These examples show how \function{get()} and \function{interpolate()}
differ.

\begin{verbatim}
>>> from ZConfig.Interpolation import get, interpolate
>>> d = {'name': 'value',
...      'top': '$middle',
...      'middle' : 'bottom'}
>>>
>>> interpolate('$name', d)
'value'
>>> interpolate('$top', d)
'$middle'
>>>
>>> get(d, 'name')
'value'
>>> get(d, 'top')
'bottom'
>>> get(d, 'missing', '$top')
'$top'
\end{verbatim}


\end{document}
